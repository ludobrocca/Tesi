% !TEX encoding = UTF-8
% !TEX TS-program = pdflatex
% !TEX root = ../tesi.tex

%**************************************************************
% Sommario
%**************************************************************
\cleardoublepage
\phantomsection
\pdfbookmark{Sommario}{Sommario}
\begingroup
\let\clearpage\relax
\let\cleardoublepage\relax
\let\cleardoublepage\relax

\chapter*{Sommario}

Il presente documento relaziona il risultato del lavoro prodotto a seguito del periodo di stage formativo svolto dal laureando Ludovico Brocca presso l'azienda SyncLab s.r.l., della durata di circa trecento ore.\\
Prima dell'inizio dell'attività di lavoro, sono stati concordati gli obiettivi con il tutor aziendale dr. Fabio Pallaro e con il relatore prof. Claudio Palazzi.\\
Tali obiettivi prevedevano un periodo iniziale di formazione su tecnologie in uso presso l'azienda e di particolare interesse personale, a cui poi è seguita l'implementazione di alcune maschere di Front-End per l'applicativo SyncRec, la cui finalità è registrare le persone richiedenti un periodo di formazione presso l'azienda.\\
Lo scopo è stato fornire un prodotto in grado di sostituire la versione precedente, ormai datata, implementando un back-end a microservizi e un front-end in Angular 8.\\
Il lavoro è stato svolto in collaborazione con altri studenti aventi il periodo di stage formativo presso l'azienda; ciò ha permesso l'adozione di metodologie AGILE/ Scrum e il raggiungimento di una maggiore comprensione della prospettiva lavorativa di un team.\\
La tesi è suddivisa in 4 capitoli:
\begin{itemize}
	\item Nel primo viene presentata l'azienda e le metodologie di lavoro in uso presso di essa;
	\item Nel secondo viene presentato il progetto e l'organizzazione generale dello stage;
	\item Il terzo prosegue analizzando nel dettaglio gli obiettivi, i requisiti e i casi d'uso dell'applicativo SyncRec; 
	\item Il quarto presenta il dettaglio della soluzione progettata e sviluppata.
	\item Il quinto presenta una relazione degli obiettivi raggiunti, insieme al loro grado di soddisfacimento e a una valutazione personale e soggettiva del lavoro svolto.
\end{itemize}
\section*{Convenzioni tipografiche}
Durante la stesura del testo sono state adottate le seguenti convenzioni tipografiche:
\begin{itemize}
	\item gli acronimi, le abbreviazioni e i termini ambigui o di uso non comune menzionati
	sono definiti nel glossario, situato alla fine del presente documento e ogni
	occorrenza è evidenziata in blu, come l'esempio seguente: \gls{Spring};
	\item i termini in lingua straniera o facenti parti del gergo tecnico sono evidenziati con
	il carattere corsivo.
\end{itemize}

%\vfill
%
%\selectlanguage{english}
%\pdfbookmark{Abstract}{Abstract}
%\chapter*{Abstract}
%
%\selectlanguage{italian}

\endgroup			

\vfill

