
%**************************************************************
% Acronimi
%**************************************************************
\renewcommand{\acronymname}{Acronimi e abbreviazioni}


\newacronym[description={\glslink{umlg}{Unified Modeling Language}}]
    {uml}{UML}{Unified Modeling Language}

%**************************************************************
% Glossario
%**************************************************************
\renewcommand{\glossaryname}{Glossario}

\newglossaryentry{incremento}
{
    name=\glslink{incremento}{Incremento},
    text=incremento,
    sort=incremento,
    description={Procedere per aggiunta ad una base, ogni incremento produce valore aggiunto rispetto al punto di arrivo precedente},
    plural=incrementi
}


\newglossaryentry{System Integrator}
{
	name=\glslink{System Integrator}{System Integrator},
	text=System Integrator,
	sort=System Integrator,
	description={
	Persona o azienda specializzata nell'integrazione fra sottosistemi, assicurando il loro corretto funzionamento},
	plural=System Integrators,
}

\newglossaryentry{iterazione}
{
    name=\glslink{iterazione}{Iterazione},
    text=iterazione,
    sort=iterazione,
    description={Procedere per rivisitazioni (può includere un incremento o addirittura un decremento).\\L'iterazione è un processo di durata non determinabile (anche potenzialmente infinita)},
    plural=iterazioni
}

\newglossaryentry{api}{
	name={API},
	description={
	si indica un insieme di procedure (in genere raggruppate per strumenti specifici) atte all'espletamento di un dato compito;
	in informatica tale concetto è concretizzato da librerie \textit{software} afferenti a un certo compito o \textit{task} da svolgere
	}
}

\newglossaryentry{JavaScript}{name={JavaScript},description={Linguaggio di scripting orientato agli oggetti e agli eventi, comunemente utilizzato nella programmazione \textit{web} lato \textit{client}}}

\newglossaryentry{Node.js}{name={Node.js},description={\textit{Runtime} di JavaScript Open-source multipiattaforma orientato agli eventi per l'esecuzione di codice JavaScript, costruita sul motore JavaScript V8 di \textit{Google Chrome}}}

\newglossaryentry{IDE}{name={IDE},description={\textit{Software} che, in fase di programmazione, supporta i programmatori nello sviluppo del codice sorgente di un programma}}

\newglossaryentry{Debugger}{name={Debugger},description={\'E un programma/software specificatamente progettato per l'analisi e l'eliminazione dei bug, ovvero errori di programmazione interni al codice di altri programmi}}

\newglossaryentry{Back-end}{name={Back-end},description={Componente aggregata che permette l'effettivo funzionamento delle interazioni tra  l'utente e l'interfaccia grafica (o \textbf{Front-end}) 
}}

\newglossaryentry{snippets}{name={Snippets},description={Frammento o esempio di codice sorgente}}

\newglossaryentry{Git}{name={Git},description={\'E un \textit{software} di controllo versione distribuito utilizzabile da interfaccia a riga di comando}}

\newglossaryentry{repository}{name={Repository},description={Un ambiente di sistema informatico condiviso, utilizzato per mantenere allineato l'avanzamento dell'applicativo nel corso del suo ciclo di vita}}

\newglossaryentry{continuos integration}{name={Continuos Integration},description={Pratica che si applica in contesti in cui lo sviluppo del \textit{software} avviene attraverso un sistema di versionamento. Consiste nell'allineamento frequente (ovvero "molte volte al giorno") dagli ambienti di lavoro degli sviluppatori verso l'ambiente condiviso}}

\newglossaryentry{branch}{name={Branch},description={Puntatore a un determinato stato di avanzamento del \textit{software} all'interno di una \textit{repository}, tipicamente vengono utilizzati per suddividere logicamente gli incrementi necessari durante il ciclo di vita di un \textit{software}}}

\newglossaryentry{merge}{name={Merge},description={Eliminazione di un \textit{branch} e conseguente allineamento degli incrementi sviluppati in un secondo \textit{branch}}}

\newglossaryentry{npm}{name={npm},description={Sigla per \textit{Node Package Manager}, è un gestore di pacchetti per il linguaggio di programmazione JavaScript}}

\newglossaryentry{web application}{name={Web Application},description={Indica un'applicazione \textit{web-based} accessibile tramite \textit{browser}}}

\newglossaryentry{framework}{name={Framework},description={Un'architettura logica di supporto su cui un software può essere progettato e realizzato}}

\newglossaryentry{Python}{name={Python},description={Python è un linguaggio di programmazione ad alto livello, orientato agli oggetti, adatto, tra gli altri usi, a sviluppare applicazioni distribuite, scripting, computazione numerica e \textit{system testing}}}

\newglossaryentry{ADA}{name={ADA},description={Sviluppato verso la fine degli anni settanta, è un linguaggio \textit{general purpose} che si presta all'utilizzo in qualsiasi dominio applicativo}}

\newglossaryentry{URL}{name={URL},description={Sigla per \textit{Uniform Resource Locator}: si tratta di una sequenza di caratteri che identifica univocamente l'indirizzo di una risorsa su una rete di computer}}

\newglossaryentry{angular CLI}{name={Angular CLI},description={Sigla per \textit{Angular Command Line Interface}, si tratta di un applicativo a riga di comando compreso nella \textit{suite} \textit{Angular} in grado di facilitare lo sviluppo di applicazioni \textit{web}}}

\newglossaryentry{DBMS}{name={DBMS},description={Sigla per \textit{Database Management System} sistema \textit{software} progettato per consentire la creazione, la manipolazione e l'interrogazione efficiente di database}}

\newglossaryentry{CRUD}{name={CRUD},description={Sigla che indica le classiche operazioni di manipolazioni di dati (\textit{Create, Read Update e Delete})}}

\newglossaryentry{RxJs}{name={RxJs},description={Libreria per il \textit{reactive programming} fortemente basata sull'\textit{Observer Pattern} }}

\newglossaryentry{web server}{name={Web Server},description={Applicazione software che, in esecuzione su un \textit{server}, è in grado di gestire le richieste di trasferimento di pagine \textit{web} di un \textit{client}, tipicamente un \textit{web browser}}}

\newglossaryentry{dependency injection}{name={Dependency Injection},description={Design pattern della Programmazione orientata agli oggetti il cui scopo è quello di semplificare lo sviluppo e migliorare la testabilità di software di grandi dimensioni}}


\newglossaryentry{open-source}{name={Open-source},description={Un software open source è reso tale per mezzo di una licenza attraverso cui i detentori dei diritti ne favoriscono la modifica, lo studio, l'utilizzo e la redistribuzione. Caratteristica principale dunque delle licenze open source è la pubblicazione del codice sorgente}}

\newglossaryentry{ng-Model}{name={ng-Model},description={L'insieme di direttive \textit{Angular} che lega vari \textit{tag} \textit{HTML} a variabili utili nella gestione di interazioni con l'utente durante l'utilizzo dell'applicazione
}}

\newglossaryentry{ISO}{name={ISO},description={Sigla per \textit{International Organization for Standardization}, si tratta della più importante organizzazione mondiale per la definizione di norme tecniche}}

\newglossaryentry{Big Data Analysis}{name={Big Data Analysis},description={Processo di raccolta e analisi di grandi volumi di dati (\textit{big data}) allo scopo di ricavarne informazioni utili. }}

\newglossaryentry{Cloud Computing}{name={Cloud Computing},description={ Indica un paradigma di erogazione di servizi offerti \textit{on demand} da un fornitore ad un cliente finale attraverso la rete Internet}}

\newglossaryentry{Internet delle Cose}{name={Internet delle Cose},description={Si riferisce all'estensione della rete internet verso oggetti e luoghi concreti}}

\newglossaryentry{Inversion of Control}{name={Inversion of Control},description={Pattern che prevede che le componenti di libreria esterne posseggano il controllo del flusso di esecuzione del programma, e che possano cedere tale controllo a componenti applicativi a loro discrezione}}

\newglossaryentry{Observer Pattern}{name={Observer Pattern},description={Pattern che sostanzialemente permette a componenti applicativi di "osservare" l'esecuzione di un altro componente e agire in conseguenza dello scatenarsi di un evento, è la base del \textit{reactive programming}}}