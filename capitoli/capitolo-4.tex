% !TEX encoding = UTF-8
% !TEX TS-program = pdflatex
% !TEX root = ../tesi.tex

%**************************************************************
\chapter{Tecnologie e progettazione}
\label{cap:tecnologie-progettazione}
%**************************************************************
IL seguente capitolo ha lo scopo di illustrare nel dettaglio le tecnologie e gli strumenti utilizzati nel corso del progetto di stage, insieme alla progettazione delle maschere richieste per il raggiungimento degli obiettivi. 

\section{Linguaggi e framework}
La seguente sezione approfondisce le tecnologie adottate(intese come \textit{framework} e linguaggi di programmazione) durante il corso del progetto.

\subsection{TypeScript}\label{typescript}
\gls{TypeScript} è un linguaggio di programmazione open-source sviluppato e mantenuto da Microsoft, esso si configura come un livello di astrazione aggiuntivo a \gls{JavaScript}, introducendo i classici paradigmi della programmazione ad oggetti, insieme alle funzionalità introdotte da ECMAScript 3 in poi .\\
Una volta compilato il codice TypeScript, esso produce in \textit{output} codice JavaScript nativo, in questo modo è garantita la compatibilità con tutti browser, \gls{Node.js}, o \textit{engine} JavaScript che supportino ECMAScript 3; in un progetto Angular, tale processo avviene in automatico tramtite esecuzione dei comandi di \gls{Angular CLI}.\\
L'utilizzo di tale linguaggio si è reso necessario una volta scelto e adottato \textit{Angular} come \textit{framework} di riferimento per lo sviluppo delle maschere di Front-End.\\
\newpage
\subsection{Angular}
\begin{figure}[!h] 
	\centering 
	\includegraphics[width=0.3\columnwidth]{immagini/angular.png} 
	\caption{Logo di Agular}
	\label{figura:angular-1}
\end{figure}
Angular è un framework sviluppato da Google utilizzato per lo sviluppo di applicazioni web responsive e compatibili con la gran parte dei dispositivi odierni, esso nasce come evoluzione \textit{AngularJS}, abbandonando l'utilizzo di JavaScript in favore di TypeScript (v. sezione \ref{typescript}), ciò rende le due versioni non compatibili.
I principali vantaggi offerti da Angular sono:
\begin{itemize}
	\item \textbf{Cross-platform:} Angular permette lo sviluppo di \textit{web application}, \textit{mobile web application}, applicazioni \textit{desktop} e applicazioni \textit{mobile};
	\item \textbf{Velocità e performance:} la struttura del \textit{framework} permette di sviluppare applicazioni veloci e dinamiche, l'elaborazione, infatti, viene eseguita quasi interamente su lato client, una volta scaricata l'applicazione dal \gls{web server}; il peso delle applicazioni, inoltre, è stato sensibilmente ridotto;
	\item \textbf{Compatibilità:} rispetto al suo predecessore, Angular semplifica l'interazione e la compatibilità con librerie esterne JavaScript, come \gls{RxJs} o \gls{immutable.js}.
\end{itemize}

In un progetto Angular, l'elemento cardine risulta essere il \textit{component}, introdotto dalla seconda versione, esso rappresenta l'elemento base della struttura gerarchica, organizzata in moduli, che permette un'efficiente suddivisione del carico di lavoro tra più programmatori, i quali non hanno necessità di conoscere interamente la logica dell'applicazione per apportare un contributo allo sviluppo.
Le seguenti sezioni hanno lo scopo di dettagliare accuratamente i principali elementi che fanno parte di un'applicazione sviluppata in Angular.


\subsubsection{Module}
I moduli rappresentano contenitori di blocchi di codice coesi afferenti a una certa funzionalità da implementare, a un \textit{workflow} da eseguire o a un insieme strettamente correlato di capacità sviluppate.
Essi possono contenere:
\begin{itemize}
	\item \textbf{Services:} classi addette alla gestione di transizioni con l'esterno o alla manipolazione dei dati (per maggiori dettagli  v. sez. \ref{service});
	\item \textbf{Routing Modules:} moduli particolari che permettono di mappare richieste \textit{URL} alla visualizzazione di \textit{component} (v. sez. \ref{routing});
	\item \textbf{Components:} elemento base di una web application sviluppata in \textit{Angular} (v. sezione \ref{component});
	\item Altri \textit{file} contenenti codice il cui \textit{scope} è definito dal modulo stesso.
\end{itemize}
Ogni applicazione sviluppata in \textit{Angular} contiene almeno un modulo, definito \textit{root module} e contenuto nel file \texttt{app.module.ts}, il quale deve illustrare i moduli e i component dichiarati.\\
La \textit{Best practice} suggerita dalla documentazione afferma che è opportuno dichiarare per ogni \textit{component} un suo modulo di appartenenza, salvo rari casi dove ciò risulterebbe ridondante (si pensi alla dichiarazione di un \textit{component} contenente una logica basilare e un \textit{template HTML} di dimensioni ridotte), in questo modo, il \textit{root module} conterrà solamente le dichiarazioni dei moduli figli, e andrà aggiornato solo nel caso in cui si renda necessaria l'aggiunta di un nuovo \textit{component}.

\subsubsection{Component}\label{component}
I \textit{component} costituiscono elementi di una web application sviluppata in \textit{Angular}, i quali assumono controllo della visualizzazione dei dati e la gestione degli eventi in determinate circostanze definite dal programmatore.\\
Ogni \textit{component} è costituito da 3 parti:
\begin{itemize}
	\item un \textit{file} \texttt{.ts} che definisce la logica sottostante per l'elaborazione dei dati;
	\item un \textit{file} \texttt{.html} che costituisce il \textit{template} per la visualizzazione su schermo del \textit{component};
	\item un \textit{file} \texttt{.css} che definisce posizione, dimensione e colore degli elementi definiti nel \textit{template}.
\end{itemize} 
Questo tipo di struttura permette una grande possibilità di riutilizzo del codice: ogni component, una volta definito, può essere visualizzato secondo quanto dichiarato nel suo modulo di appartenenza. In questo modo, nel caso in cui fosse necessario riutilizzare un \textit{component} precedentemente definito, sarà sufficiente importare il suo modulo di riferimento ove desiderato.

\subsubsection{Service}\label{service}
\textit{Angular} ha predisposto una struttura facente uso della \gls{dependency injection}, la quale permette di iniettare dipendenze nei \textit{component} senza pregiudicare la manutenibilità del codice; applicando questo approccio è possibile dividere chiaramente la visualizzazione dei dati sullo schermo dal suo relativo recupero.\\
L'elemento che si occupa del \textit{fetch} dei dati in una applicazione \textit{Angular} è il \textit{service}: la documentazione suggerisce di rendere un \textit{service \textbf{Injectable}}, in questo modo è sufficiente che il \textit{component} dichiari la dipendenza nel suo costruttore per poterlo utilizzare.
Un'approccio di questo tipo ha l'ulteriore vantaggio di rendere molto più pulita la gestione degli errori nelle transazioni, che è interamente adibita ai service facenti le chiamate.

\subsubsection{Routing}\label{routing}
Il \textit{Routing} costituisce il meccanismo implementato da \textit{Angular} per la navigazione all'interno di una \gls{web application}.\\
Normalmente un utente per navigare all'interno di un sito tramite \textit{browser} effettua una tra queste tre azioni:
\begin{itemize}
	\item Digita un URL nella barra degli indirizzi e naviga direttamente alla pagina desiderata;
	\item Clicca un \textit{link} all'interno della pagina e il browser effettua la navigazione;
	\item Usa i pulsanti di \textit{back} e \textit{forward} per visualizzare pagine precedentemente visitate.
\end{itemize}
Il \textit{Router} di \textit{Angular}, utlizzando questo modello, è in grado di mappare le richieste \textit{URL} a delle \textit{view} generate dal client, con l'ulteriore possibilità di inserimento di parametri opzionali per la visualizzazione di determinati dati.\\
Nel caso del progetto SyncRec, il \textit{Routing} tramite parametri opzionali si è rivelato indispensabile per la visualizzazione del dettaglio di una persona.
\paragraph*{Funzionamento e implementazione}
Il meccanismo di implementazione del \textit{Routing} è molto simile a quello dei moduli: ogni \textit{component} deve essere corredato da un \textit{Routing Module} che definisca il \textit{path} e il corrispettivo \textit{component} collegato, successivamente, ogni \textit{Routing Module} deve essere importato e dichiarato nel corrispettivo modulo di appartenenza, in modo che il \textit{root module} possa essere a conoscenza della mappatura completa degli \gls{URL} con i component associati.

\subsection{Npm}
\textit{Npm} (Node Package Manager) si configura come il gestore di pacchetti predefinito per l'ambiente \textit{runtime} di \textit{JavaScript} \gls{NodeJs}; il suo utlizzo si è rivelato necessario per l'inclusione di alcune dipendenze necessarie per il completamento del progetto (\gls{angular CLI} primo fra tutti).\\
I pacchetti sono memorizzati in un database remoto chiamato \textit{npm registry} e l'aggiunta, rimozione e aggiornamento degli stessi avviene tramite il \textit{client} a riga di comando messo a disposizione, chiamato anch'esso \textit{npm}.

\subsection{Bootstrap}
\textit{Bootstrap} è una raccolta di strumenti \gls{open-source} per lo sviluppo di siti e applicazioni web. Essa contiene moduli basati su \textit{HTML} e \textit{CSS} per la creazione dei componenti facenti parte di un'interfaccia (moduli, pulsanti e navigazione), insieme all'integrazione con alcune componenti opzionali di \textit{JavaScript}.\\
Si tratta di un \textit{toolkit} compatibile con tutti i browser moderni e facente ampio uso del \textit{responsive web design}, è stato integrato nel progetto allo scopo di applicare rapidamente uno stile comune agli elementi facenti parte dell'applicazione.
\bigbreak
\noindent
Per ulteriori informazioni sulle tecnologie apprese nel corso del periodo di stage v. sezione \ref{Appendice-1}


\subsection{HTML e CSS}
L'utilizzo di HTML e CSS si è reso necessario una volta scelto \textit{Agular} come \gls{framework} di riferimento; essi sono linguaggi consolidati che permettono la definizione della struttura delle pagine web (nel caso di Angular essi permettono di definire la struttura dei singoli component).

\subsection{Java}
Java è il linguaggio di programmazione utilizzato da \gls{Spring}, il \gls{framework} di riferimento per lo sviluppo del \textit{back-end} dell'applicativo \textit{SyncRec}.

\subsection{Spring}
\begin{figure}[!h] 
	\centering 
	\includegraphics[width=0.4\columnwidth]{immagini/spring.png} 
	\caption{Logo di Spring}
	\label{figura:spring-1}
\end{figure}
\gls{Spring} è un framework scritto in Java utilizzato per lo sviluppo di applicazioni \textit{enterprise}; esso è sudddiviso in moduli che rispondono a vari tipi di esigenze, in modo da poter includere nella propria applicazione solo le funzionalità desiderate.\\
\gls{Spring} è fortemente basato sul principio \gls{Inversion of Control}, che delega la maggior parte del controllo sul flusso di esecuzione al \textit{framework} stesso, in modo da ridurre il più possibile l'introduzione di errori.


\subsection{MongoDb}
\textit{MongoDB} è un \gls{DBMS} non relazionale orientato ai documenti scritti in \textit{JSON}.\\
Si tratta di un software estremamente diffuso e fortemente scalabile, cosa che ha permesso il suo utilizzo anche in ambienti enterprise con milioni di transazioni giornaliere.\\
Come affermato in precedenza, esso è stato utilizzato come DBMS di riferimento all'interno del progetto di stage \textit{SyncRec}, con risultati estremamente positivi.


\subsubsection{Spring Core}
Detto anche \textit{core container}, esso rappresenta la parte principale di Spring,e tutto il framework è costruito sopra di esso.
Insieme al modulo Beans è responsabile della funzionalità di IoC (precedentemente descritta) e  di Dependency Injection, introdotta tramite \textit{getters, setters, factory methods} e costruttori.
Ciò permette di concretizzare i due principi, delegando al \textit{framework} il compito di iniettare le dipendenze del \textit{container}.



\subsubsection{Spring Boot}
Costituisce il modulo necessario per l'esecuzione di applicazioni \textit{Spring}, spesso esso non necessita di particolari configurazioni, e risulta molto facile includere altre librerie di terze parti e moduli \textit{Spring} aggiuntivi.\\

\subsubsection{Spring Data}
Costituisce il modulo di riferimento per la lettura e scrittura dei dati in applicazioni \textit{Spring}, vi sono varie diramazioni 	a seconda della tecnologia utilizzata (ad esempio JDBC, MongoDB, Apache); uno dei moduli più importanti approfonditi durante il periodo di stage risulta essere \textit{Spring Data Rest}, che permette la creazione di un'applicazione a microservizi molto facilmente, tramite le tipiche \textit{annotations} di \textit{Spring} (definite tramite l'operatore \@).

\section{Ambiente di Sviluppo}
La sezione seguente illustra gli strumenti facenti parte dell'ambiente di sviluppo e utilizzati nel corso del progetto.

\subsection{Sistema operativo}
I sistemi operativi adottati per ilcompletamento del progetto sono due:
\begin{itemize}
	\item Ubuntu KDE 2019
	\item Ubuntu 18.04.2 LTS
\end{itemize}

\subsection{IDE}
L'adozione di un corretto ambiente di sviluppo integrato (\textit{Integrated development environment}, spesso abbreviato con \textbf{IDE}), è di primaria importanza per il corretto sviluppo di un \textit{software}, la scelta è spesso soggettiva e determinata dai giusti personali del programmatore.
In questo progetto, viene adottato \textit{Visual Studio Code}, una \textit{IDE} \gls{open-source} sviluppata da Microsoft.\\
Si tratta di un \textit{software} molto leggero e adattabile alle proprie esigenze tramite l'aggiunta di estensioni; queste due qualità rendono \textit{VsCode} il software ideale per lo sviluppo di un applicativo in \textit{Angular}.
Alcuni dei principali vantaggi di \textit{VsCode sono}
\begin{itemize}
	\item Piena integrazione con \gls{Git};
	\item Ampia gamma di estensioni;
	\item Introduzione di \textbf{\textit{IntelliSense}}, software per l'autocompletamento del codice
	\item Personalizzabile.
\end{itemize}
Secondo un sodaggio effettuato da \textit{StckOverflow} nel 2019, \textit{VsCode} risulta l'ambiente di sviluppo integrato più utilizzato dai programmatori, con il 50.7\% di scelta su 87.317 partecipanti.


\paragraph{Estensioni}
\begin{itemize}
	\item \textbf{Angular Snippets 8:} Permette l'aggiunta di \gls{snippets} per \textit{Angular} e \textit{HTML};
	\item \textbf{angular2-inline:} Per l'autocompletamento e l'evidenziazione dei \textit{template inline} che possono essere utilizzati nei \textit{component};
	\item \textbf{Debugger for Chrome:} Questa estensione, sviluppata anch'essa da Microsoft, permette di utilizzare un \gls{Debugger} per JavaScript/TypeScript;
	\item \textbf{Material Icon Theme:} Permette di visualizzare immediatamente l'estensione di un \textit{file} tramite l'utilizzo di un'icona a fianco del nome;
	\item \textbf{npm:} Permette l'esecuzione rapida di script tramite \gls{npm};
	\item \textbf{TSLint:} Fornisce supporto per l'utilizzo di \textit{TSLint} all'interno di \textit{Visual Studio Code} (v. sezione \ref{tslint}) ;
\end{itemize}

\subsection{Versionamento}
Lo strumento di versionamento utilizzato è \gls{Git} e il codice viene memorizzato in una \gls{repository} remota presso \textit{GitLab}. La repository è privata, e l'accesso è stato consentito dal tutor aziendale dr Fabio Pallaro.\\
La necessità di lavorare in gruppo, inoltre, richiede un processo di \gls{continuos integration} rapido ed efficiente: a tal scopo, viene integrata l'estensione \textit{git-flow}  al normale strumento di versionamento. Essa permette la creazione immediata di \gls{branch} di \textit{feature} separate, il successivo \gls{merge} è altresì possibile mediante l'utilizzo di un singolo comando.\\
La necessità di venire incontro a casistiche particolari che possono incorrere nella gestione della \textit{repository} ha richiesto la scelta di una \textins{IDE} che possa svolgere tale compito. Il \textit{software} adottato per questo progetto è \textit{GitKraken}, un applicativo \textit{open-source} che fornisce  supporto alla gestione di repository \gls{Git}, insieme all'integrazione dell'estensione \textit{git-flow}.

\subsection{Testing}
Viene adottato \textit{Karma} come \textit{Test runner} all'interno del progetto, come suggerito dalla documentazione di \textit{Angular}
I test vengono codificati in \textit{file} con estensione \texttt{.spec.ts} (creati automaticamente da Angular CLI al momento dell'inserimento di un component) ed eseguiti tramite il comando \texttt{npm test}; la configurazione di \textit{Karma}, invece, avviene tramite un \textit{file} \texttt{karma.conf.js}.

\subsection{Back-end}
I microservizi sono installati in una macchina di proprietà di SyncLab, e sono accedibili tramite un URL dalle persone autorizzate.\\
L'integrazione, quindi, è avvenuta presso l'azienda, in quanto non è possibile accedere al \gls{Back-end} da dispositivi esterni.

\subsection{Analisi statica del codice} \label{tslint}
Viene utilizzato TSLint come strumento di analisi statica del codice, allo scopo di aumentare la leggibilità, manutenibilità e prevenzione di errori funzionali all'interno del codice, la sua configurazione avviene tramite un \textit{file} \texttt{tslint.json} e gli errori sono immediatamente visibili al momento della scrittura.

\subsection{Altri strumenti utilizzati}
Per effettuare richieste al \gls{Back-end}, viene utilizzato \textit{Postman}, che permette di testare il corretto funzionamento di un microservizio tramite le classiche chiamate \textit{GET, PUT, POST, DELETE}; in questo modo si può verficare prima della scritttura del codice che il comportamento del \gls{Back-end} sia conforme rispetto ai risultati attesi.

