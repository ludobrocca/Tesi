% !TEX encoding = UTF-8
% !TEX TS-program = pdflatex
% !TEX root = ../tesi.tex

%**************************************************************
\chapter{Introduzione}
\label{cap:introduzione}
%**************************************************************

%Introduzione al contesto applicativo.\\

%\noindent Esempio di utilizzo di un termine nel glossario \\
%\gls{api}. \\

%\noindent Esempio di citazione in linea \\
%\cite{site:agile-manifesto}. \\

%\noindent Esempio di citazione nel pie' di pagina \\


%**************************************************************
\section{L'azienda: SyncLab s.r.l.}
SyncLab nasce nel 2002 come Software house, trasformandosi poi nella figura di \gls{System Integrator} in conseguenza del rapido ampliamento delle competenze e del dominio tecnologico offerto; oggi SyncLab vanta un organico di oltre 200 persone e 4 sedi sul territorio italiano (Napoli, Roma, Padova e Milano).\\


%**************************************************************
\subsection{Principali aree applicative}
SyncLab offre offre i propri servizi operando su diverse aree applicative di primaria importanza, ponendo l'accento sulla predisposizione di un ambiente di lavoro collaborativo ed efficace, allo scopo di fornire prodotti di qualità che rispecchino gli standard adottati.\\
\'E di primaria importanza per SyncLab, infatti, fornire ai propri clienti la possibilità di adottare il miglior software \gls{ERP} (Enterprise Resource Planning) possibile, valutando accuratamente caso per caso le necessità e i bisogni, ponendosi come partner di supporto per lo sviluppo e la crescita dei clienti.\\
L'azienda, inoltre, offre ai propri clienti la possibilità di usufruire dell'\textit{expertise} accumulata nel corso del suo operato nel campo della \gls{ICT} (Information and Comunications Technology), fornendo consulenza informatica \textit{ad hoc}, sviluppo di sistemi distribuiti \textit{enterprise} supportati dagli standard ISO più diffusi (ISO 9001, ISO 14001, ISO 27001) e fornitura di servizi comprendenti la \textit{Business Consultancy}, \textit{It Consultancy} e \textit{Project Financing}.


\subsection{Prodotti e soluzioni}
\begin{itemize}
	\item \textbf{E-Health:} uno  dei principali prodotti sviluppati da SyncLab è SynClinic, un sitema informativo sanitario che offre la possibilità di gestire tutti i processi clinici e amministrativi di ospedali, cliniche e case di cura.
	\item \textbf{Privacy e Sicurezza:} nel corso del suo operato nell'ambito della privacy e sicurezza, l'azienda ha sviluppato diversi applicativi: i principali sono DPS 4.0 (per la gestione delle informazioni personali in seguito al GDPR per la protezione dei dati) e StreamLog, al fine di monitorare gli accessi degli utenti ai propri sistemi informativi.
	\item \textbf{Big Data e Mobile development:} uno dei prodotti di punta di SyncLab è StreamCrusher, un applicativo in grado di raccogliere e unificare i dati generati da un'azienda, individuando rapidamente criticità e vulnerabilità, aiutando le aziende nel processo di \textit{decision making}.
\end{itemize}

\subsection{Ricerca e Svilupppo}
Laboratorio
%**************************************************************
