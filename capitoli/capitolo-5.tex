% !TEX encoding = UTF-8
% !TEX TS-program = pdflatex
% !TEX root = ../tesi.tex

%**************************************************************
\chapter{Progettazione e codifica}
\label{cap:progettazione-codifica}
%**************************************************************

La seguente sezione ha lo scopo di illustrare la soluzione ideata dal laureando per il soddisfacimento dei requisiti concordati con il tutor aziendale descritti nel capitolo \ref{cap:descrizione-stage}. Nella prima sezione viene presentato l'applicativo con il dovuto corredo di immagini, mentre nel secondo viene descritta accuratamente l'architettura implementata.


\section{L'applicativo SyncRec}
\subsection{Struttura e sviluppo del progetto}
Come già accennato in precedenza, \textit{Angular} suggerisce di applicare una forte struttura modulare alle proprie \textit{web-application}, tale approccio è stato nella gran parte perseguito, con alcune eccezioni per determinati elementi condivisi fra più \textit{component}; un'applicazione di tale struttura, inoltre, permette lo sviluppo delle proprie componenti senza dover necessariamente conoscere l'intera architettura nel dettaglio.

\begin{figure}[!h] 
	\centering 
	\includegraphics[width=1\columnwidth]{immagini/usecase/UML1} 
	\caption{Struttura dei component sviluppati per l'applicativo SyncRec}
	\label{figura:UML1}
\end{figure}

La figura \ref{figura:UML1} illustra la struttura dei component sviluppati per l'applicativo SyncRec, a scopo di sintesi sono state omesse alcune parti proprie del \textit{framework} di riferimento (come i moduli e gli \textit{asset}) o sviluppate da altri studenti nel corso del progetto.\\
I moduli \textit{applicant-list}, \textit{applicant-detail} e \textit{viewskillmatrix} rappresentano i componenti grafici dell'applicazione, il modulo \textit{shared} rappresenta l'insieme di \textit{component} condivisi nel \textit{namespace} globale e il modulo \textit{core} contiene i \textit{service}; quest'ultimi hanno il duplice scopo di recuperare i dati dal \textit{Back-end} e di fornire alcuni metodi di utilità necessari a svolgere determinate operazioni (come l'ordinamento o l'emissione di eventi verso \textit{components} padri).\\
La figura \ref{figura:UML2} illustra come il modulo \textit{service} si occupa di dialogare con il \textit{Back-end} scritto in \textit{Spring}.\\
In ambito di progettazione, dialogando con gli altri laureandi che hanno preso parte alla realizzazion di SyncRec, sono definite alcune linee guida da perseguire nel corso della codifica.\\
Tali linee guida sono:
\begin{itemize}
	\item Utilizzare il più possibile le direttive \textit{Angular} (come \gls{ng-Model}) all'interno del \textit{HTML} dei \textit{component}.
	\item Definire per ciascun \textit{component} il relativo modulo di referimento, tale prassi è fortemente consigliata dalla documentazione di \textit{Angular}.
	\item Per quanto riguarda il \textit{CSS}, favorire l'utilizzo della versione 3.0 piuttosto che della libreria esterna \textit{bootstrap}.
\end{itemize}

\begin{figure}[!h] 
	\centering 
	\includegraphics[width=1\columnwidth]{immagini/usecase/UML2} 
	\caption{Integrazione tra Front-end e Back-end}
	\label{figura:UML2}
\end{figure}

\subsection{Homepage}
La homepage dell'applicativo (v. figura \ref{figura:homepage}) si presenta come un semplice \textit{menù} con 4 voci.
\begin{itemize}
	\item \textbf{Aggiungi persone} reindirizza verso il \textit{component} adibito all'aggiunta degli \textit{applicant};
	\item \textbf{Visualizza persone} reindirizza verso il \textit{component} adibito alla visualizzazione della lista degli \textit{applicant}, sviluppato dal laureando nel corso del progetto di stage;
	\item \textbf{Ricerca Persone} reindirizza verso il component adibito alla ricerca delle persone;
	\item \textbf{Gestione della configurazione} reindirizza verso il component adibito alla gestione della configurazione dell'applicativo.
\end{itemize}
In aggiunta, è possibile effettuare il \textit{logout} con un pulsante posto in alto, vicino al logo.
\vspace{0.5em}
\begin{figure}[!h] 
	\centering 
	\includegraphics[width=1\columnwidth]{immagini/svil/homepage} 
	\caption{Schermata della homepage}
	\label{figura:homepage}
\end{figure}

\subsection{Maschera della visualizzazione degli\applicant} \label{section: applicant-list}
La fugura \ref{figura:lista}, mostra il \textit{component} sviluppato per la visualizzazione degli \textit{applicant}, per ciascun \textit{applicant} viene visualizzato il Cognome, Nome e l'email, insieme a un tasto per l'eliminazione e uno per la visualizzazione del dettaglio della persona, il quale reindirizza verso la maschera descritta nella sezione \ref{CRUD}.\\
Per realizzare questa \textit{task}, è stato necessario reperire le informazione tramite un service definito appositamente, il quale effettua una semplice GET al relativo microservizio sviluppato nel Back-end (v. codice \ref{get-applicant}).
Successivamente, i dati venivano memorizzati in un apposito array e visualizzati tramite una direttiva angular chiamata \textit{ng-for} (v.codice  \ref{ng-for}).

\vspace{0.5em}
\begin{figure}[!h] 
	\centering 
	\includegraphics[width=1\columnwidth]{immagini/svil/lista} 
	\caption{Schermata della visualizzazione lista degli applicant}
	\label{figura:lista}
\end{figure}

\begin{lstlisting}[label=get-applicant,caption=Funzione del service che effettua la chiamata GET]
getAllApplicants(): Observable<Applicant[]> {
	return this.http.get<Applicant[]>(this.baseUrl).pipe(
		catchError(this.handleError)
	);
}
\end{lstlisting} 

\begin{lstlisting}[label=ng-for,caption=Visualizzazione degli applicant nel codice HTML]
<tr class="hover-tr" *ngFor="let appl of filteredApplicants;" >
	<td>
		{{ appl.surname }}
	</td>
	<td>
		{{ appl.name }}
	</td>
	<td>
		{{ appl.email }}
	</td>
	<td>
	<button class="btn" id="btn-delete-applicant-list"
		(click)="deleteApplicant(appl.id)">Elimina</button>
	</td>
	<td>
	<span class="immagineEsterna" (click)="routerRedirect(appl.id)"><i class="fa fa-search fa-2x"></i></span>
	</td>
</tr>
\end{lstlisting} 

Posti sopra la tabella, inoltre, è possibile visualizzare un campo di testo, che permette di filtrare i risultati secondo il nome, cognome o email; il pulsante posto in alto a destra invece visualizza il component adibito all'applicazione di filtri aggiuntivi sul totale degli applicant, come si può vedere in figura \ref{figura:filtri}.\\
Il box di testo non è altro che un \textit{component} condiviso in grado di emettere eventi al component padre, e filtrare i dati di conseguenza.\\
Il filtro dinamico è stato ottenuto tramite l'azione congiunta di un service (v codice. \ref{event-e}) e di un \textit{component} costituito da un singolo campo di testo; il \textit{component} padre in questo modo può applicare il filtro sugli applicant ogni qualvolta il contenuto del campo di testo viene cambiato, in modo simile a quanto accade con l'\gls{Observer Pattern}.

\begin{lstlisting} [label=event-e, caption= Event-Emitter service]
export class EventEmitterService {

	invokeFirstComponentFunction = new EventEmitter();
	// variabile per la sottoscrizione di un component per la reazione // ad eventi lanciati
	subsVar: Subscription;

	constructor() { }

	onComponentAction() {
		this.invokeFirstComponentFunction.emit();
	}
}
\end{lstlisting}

\begin{lstlisting} [label=filter, caption= Funzione che filtra i dati degli applicant]
filter(data: string) {
	if (data) {
		this.filteredApplicants = this._applicants.filter((applicant: Applicant) => {
			return applicant.name.toLowerCase().indexOf(data.toLowerCase()) > -1 ||
			applicant.surname.toLowerCase().indexOf(data.toLowerCase()) > -1 ||
			applicant.email.toLowerCase().indexOf(data.toLowerCase()) > -1;
		});
		} else {
		this.filteredApplicants = this._applicants;
	}
}

\end{lstlisting}

La visualizzazione del \textit{component} relativo all'aggiunta di filtri aggiuntivi viene gestita tramite una semplice direttiva \textit{angular} chiamata \textit{Ng-if}, la quale si lega a una variabile booleana e determina la visualizzazione o meno del \textit{component} a seconda del valore assegnato. La selezione dei valori, invece, viene gestita tramite i \textit{form} angular, i quali legano i valori contenuti nei vari \textit{widget} a delle strutture che gestiscono autonomamente valori di default, validazione, \textit{submit} dei risultati e quant'altro, rendendo la struttura molto solida, sicura e poco soggetta a errori.
Nel codice è possibile vedere l'inizializzazione del \textit{FormBuilder}, la sopracitata struttura in grado di gestire i dati di un \textit{form}, notare che viene inserito un pattern per la validazione ove necessario.
\newpage
\begin{lstlisting}[label= form, caption= Inizializzazione di un FormBuilder]


this.filterGroup = this.formBuilder.group({
	sesso: [''],
	name: ['',  Validators.pattern('^[a-zA-Z]*$')],
	surname: ['',  Validators.pattern('^[a-zA-Z]*$')],
	qualification: [''],
	seniority: [''],
	ambito: [''],
	geodisp: this.formBuilder.array([]),
	skill: [''],
	skillLevel: [''],
	scartato: [false],
});

\end{lstlisting}

\vspace{0.5em}
\begin{figure}[!h] 
	\centering 
	\includegraphics[width=1\columnwidth]{immagini/svil/filtri} 
	\caption{Schermata della selezione dei filtri da applicare alla lista degli applicant}
	\label{figura:filtri}
\end{figure}

\subsection{Maschera delle operazioni CRUD su un\applicant}\label{CRUD}
\vspace{0.5em}
\begin{figure}[!h] 
	\centering 
	\includegraphics[width=1\columnwidth]{immagini/svil/applicant}
	\caption{Maschera CRUD del singolo applicant}
	\label{figura:applicant}
\end{figure}
La gestione delle operazioni \gls{CRUD} di un \textit{applicant} avviene tramite la maschera visible nella figura \ref{figura:applicant}; l'approccio adottato è molto simile a quanto avviene con la gestione della maschera per la selezione dei filtri aggiuntivi descritta nella sezione \ref{section: applicant-list}: viene utilizzato un \textit{FormBuilder} per la gestione dei singoli valori di un \textit{applicant}, una volta modificati, un controllo su un parametro dei \textit{FormControl}, chiamato \texttt{touched}, permette la visualizzazione dei tasti di conferma o annullamento delle modifiche, in caso di conferma viene effettuata una chiamata \textit{PUT} tramite l'apposito servizio (descritta nel codice TODO), in caso contrario vengono ripristinati i valori antecedenti alle modifiche apportate dall'utente.\\
\newpage
\begin{lstlisting}[label=PUT-call, caption=chiamata PUT al microservizio del Back-end di SyncRec]
putApplicant(applicant: Applicant): Observable<any> {
	return this.http.patch(this.baseUrl + '/' + applicant.id, applicant).pipe(
		catchError((err: HttpErrorResponse) => {
			if (err.error instanceof Error) {
				// A client-side or network error occurred.
				const details = {detail: err.error, status: err.status};
				return throwError(details);
			} else {
				// The backend returned an unsuccessful response code.
				const details = {detail: err.error, status: err.status};
				return throwError(details);
			}
		})
	);
	
}
\end{lstlisting}
Con i pulsanti posti in basso è possibile tornare indietro, eliminare un \textit{applicant} (tramite una chiamata DELETE al microservizio di Back-end), o visualizzare l'insieme di competenze possedute tramite l'apposita maschera descritta nella sezione \ref{section:skillmatrix}
\vspace{0.5em}.

\subsubsection{Maschera di visualizzazione di uno skillmatrix}\label{section:skillmatrix}
\begin{figure}[!h] 
	\centering 
	\includegraphics[width=1\columnwidth]{immagini/svil/skillmatrix} 
	\caption{Maschera CRUD dello skillmatrix}
	\label{figura:skillmatrix}
\end{figure}
