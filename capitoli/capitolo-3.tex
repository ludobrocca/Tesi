% !TEX encoding = UTF-8
% !TEX TS-program = pdflatex
% !TEX root = ../tesi.tex

%**************************************************************
\chapter{Analisi dei requisiti}
\label{cap:descrizione-stage}
%**************************************************************
In questa sezione vengono illustrati i requisiti individuati per il progetto SyncRec.

\section{Tipologie di utenti}
Questa sezione ha lo scopo di dettagliare gli attori individuati; ciascun attore si differenzia dagli altri per i permessi di accesso a determinate parti del sistema:
\begin{itemize}	
	\item \textbf{\nonlogged:} utente che non ha eseguito il \textit{login} presso l'applicazione;
	\item \textbf{\loggedusr:} utente che ha effettuato l'accesso alla piattaforma SyncRec;
	\item \textbf{\applicant:} utente a cui è stato inviato un link specifico per la compilazione dello skillmatrix prima dello svolgimento di un colloquio lavorativo. 
\end{itemize}

\section{Casi d'uso}
La sezione presenta i casi d'uso individuati per l'intero progetto SyncRec.

%\subsection{UC-}
%\begin{itemize}
%\item \textbf{Attori:}
%\item \textbf{Precondizione:}
%\item \textbf{Scenario Principale:}
%\begin{enumerate}
%	\item 
%\end{enumerate}
%\item \textbf{PostCondizione:}
%\item \textbf{Estensioni:}
%\end{itemize}


\subsection{UC-1 Visualizzazione vista degli\applicant}
\begin{itemize}
	\item \textbf{Attori:}\loggedusr
	\item \textbf{Precondizione:} L'utente si trova nella homepage dell'applicazione.
	\item \textbf{Scenario Principale:}
		\begin{enumerate}
			\item L'utente clicca sul pulsante "Visualizza lista delle persone" 
		\end{enumerate}
	\item \textbf{PostCondizione:} l'utente accede alla vista di visualizzazione lista degli applicant.
\end{itemize}

\subsection{UC-2: Visualizzazione lista degli\applicant}
\begin{itemize}
	\item \textbf{Attori:}\loggedusr
	\item \textbf{Precondizione:} l'utente si trova nella vista di visualizzazione lista degli applicant.
	\item \textbf{Scenario Principale:} 
		\begin{enumerate}
			\item l'utente visualizza il cognome dell'applicant; 
			\item l'utente visualizza il nome dell'applicant;
			\item l'utente visualizza l'email dell'applicant;
		\end{enumerate}
	\item \textbf{PostCondizione:} l'utente si trova nella vista di visualizzazione lista degli applicant.
\end{itemize}

\subsection{UC-3 Aggiunta filtro su nome/cognome su lista degli\applicant}
\begin{itemize}
	\item \textbf{Attori:}\loggedusr
	\item \textbf{Precondizione:} l'utente si trova nella vista di visualizzazione lista degli applicant.
	\item \textbf{Scenario Principale:}
		\begin{enumerate}
			\item l'utente inserisce nel campo di testo un carattere o un insieme di caratteri.
		\end{enumerate}
	\item \textbf{PostCondizione:} viene visualizzata la lista degli applicant dove nome e cognome contengono i caratteri inseriti
\end{itemize}

\subsection{UC-4 Eliminazione di un\applicant}
\begin{itemize}
\item \textbf{Attori:}\loggedusr
\item \textbf{Precondizione:} l'utente si trova nella vista di visualizzazione lista degli applicant.
\item \textbf{Scenario Principale:}
\begin{enumerate}
	\item l'utente clicca il pulsante "Elimina" riferito a un certo applicant
\end{enumerate}
\item \textbf{PostCondizione:} Viene visualizzato un messaggio di richiesta di conferma.
\end{itemize}


\subsection{UC-4.1 Conferma eliminazione di un\applicant}
\begin{itemize}
\item \textbf{Attori:}\loggedusr
\item \textbf{Precondizione:}l'utente si trova nella vista di visualizzazione lista degli applicant e ha richiesto l'eliminazione di un applicant.
\item \textbf{Scenario Principale:}
\begin{enumerate}
	\item L'utente visualizza il messaggio di richiesta di conferma;
	\item L'utente preme il tasto "Ok";
\end{enumerate}
\item \textbf{PostCondizione:} si chiude il messaggio di conferma, avviene l'eliminazione dell'applicant e il \textit{refresh} della pagina;
\item \textbf{Estensioni:} 
\begin{enumerate}
	\item l'utente preme il tasto "Annulla"
	\item il messaggio di conferma viene chiuso.
\end{enumerate}
\end{itemize}

\subsection{UC-5 visualizzazione vista dei filtri}
\begin{itemize}
\item \textbf{Attori:}\loggedusr
\item \textbf{Precondizione:} l'utente si trova nella vista di visualizzazione lista degli applicant.
\item \textbf{Scenario Principale:}
\begin{enumerate}
	\item l'utente clicca sul tasto "Applica filtri aggiuntivi"
\end{enumerate}
\item \textbf{PostCondizione:} l'utente visualizza la vista relativa ai filtri
\end{itemize}

\subsection{UC-7 Aggiunta dei filtri}
\begin{itemize}
\item \textbf{Attori:}\loggedusr
\item \textbf{Precondizione:} l'utente si trova nella vista relativa ai filtri.
\item \textbf{Scenario Principale:}
\begin{enumerate}
	\item l'utente aggiunge un filtro mediante l'inserimento di un input
\end{enumerate}
\item \textbf{PostCondizione:} l'utentesi trova nella vista relativa ai filtri e visualizza l'input immesso sullo schermo.
\end{itemize}

\subsection{UC7.1- Aggiunta di un filtro per il nome}
\begin{itemize}
\item \textbf{Attori:}\loggedusr
\item \textbf{Precondizione:} l'utente si trova nella vista relativa ai filtri.
\item \textbf{Scenario Principale:}
\begin{enumerate}
	\item l'utente inserisce una parola chiave in un campo di testo dedicato al nome
\end{enumerate}
\item \textbf{PostCondizione:}  l'utente si trova nella vista relativa ai filtri e visualizza l'input immesso sullo schermo.
\end{itemize}

\subsection{UC7.2- Aggiunta di un filtro per il cognome}
\begin{itemize}
	\item \textbf{Attori:}\loggedusr
	\item \textbf{Precondizione:} l'utente si trova nella vista relativa ai filtri.
	\item \textbf{Scenario Principale:}
	\begin{enumerate}
		\item l'utente inserisce una parola chiave in un campo di testo dedicato al cognome
	\end{enumerate}
	\item \textbf{PostCondizione:}  l'utente si trova nella vista relativa ai filtri e visualizza l'input immesso sullo schermo.
\end{itemize}

\subsection{UC7.3- Aggiunta di un filtro per il genere}
\begin{itemize}
	\item \textbf{Attori:}\loggedusr
	\item \textbf{Precondizione:} l'utente si trova nella vista relativa ai filtri.
	\item \textbf{Scenario Principale:}
	\begin{enumerate}
		\item l'utente seleziona da un menù a tendina il genere desiderato
	\end{enumerate}
	\item \textbf{PostCondizione:}  l'utente si trova nella vista relativa ai filtri e visualizza l'opzione selezionata sullo schermo.
\end{itemize}

\subsection{UC7.4- Aggiunta di un filtro per il titolo di studio}
\begin{itemize}
	\item \textbf{Attori:}\loggedusr
	\item \textbf{Precondizione:} l'utente si trova nella vista relativa ai filtri.
	\item \textbf{Scenario Principale:}
	\begin{enumerate}
		\item l'utente seleziona da un menù a tendina il titolo di studio desiderato
	\end{enumerate}
	\item \textbf{PostCondizione:}  l'utente si trova nella vista relativa ai filtri e visualizza l'opzione selezionata sullo schermo.
\end{itemize}
\subsection{UC7.5- Aggiunta di un filtro per l'ambito lavorativo}
\begin{itemize}
	\item \textbf{Attori:}\loggedusr
	\item \textbf{Precondizione:} l'utente si trova nella vista relativa ai filtri.
	\item \textbf{Scenario Principale:}
	\begin{enumerate}
		\item l'utente seleziona da un menù a tendina l'ambito lavorativo desiderato
	\end{enumerate}
	\item \textbf{PostCondizione:}  l'utente si trova nella vista relativa ai filtri e visualizza l'opzione selezionata sullo schermo.
\end{itemize}

\subsection{UC7.6- Aggiunta di un filtro relativo ad una skill }
\begin{itemize}
	\item \textbf{Attori:}\loggedusr
	\item \textbf{Precondizione:} l'utente si trova nella vista relativa ai filtri.
	\item \textbf{Scenario Principale:}
	\begin{enumerate}
		\item l'utente seleziona da un menù a tendina il nome della skill desiderato
	\end{enumerate}
	\item \textbf{PostCondizione:}  l'utente si trova nella vista relativa ai filtri e visualizza l'opzione selezionata sullo schermo.
\end{itemize}

\subsection{UC7.7- Aggiunta di un filtro relativo al livello di una skill}
\begin{itemize}
	\item \textbf{Attori:}\loggedusr
	\item \textbf{Precondizione:} l'utente si trova nella vista relativa ai filtri.
	\item \textbf{Scenario Principale:}
	\begin{enumerate}
		\item l'utente seleziona da un menù a tendina un valore da 1 a 4 rappresentante il livello della skill desiderato.
	\end{enumerate}
	\item \textbf{PostCondizione:}  l'utente si trova nella vista relativa ai filtri e visualizza l'opzione selezionata sullo schermo.
\end{itemize}

\subsection{UC7.8- Aggiunta di un filtro per il genere}
\begin{itemize}
	\item \textbf{Attori:}\loggedusr
	\item \textbf{Precondizione:} l'utente si trova nella vista relativa ai filtri.
	\item \textbf{Scenario Principale:}
	\begin{enumerate}
		\item l'utente seleziona da un menù a tendina il genere desiderato
	\end{enumerate}
	\item \textbf{PostCondizione:}  l'utente si trova nella vista relativa ai filtri e visualizza l'opzione selezionata sullo schermo.
\end{itemize}

\subsection{UC7.9- Aggiunta di un filtro per la disponibilità geografica}
\begin{itemize}
	\item \textbf{Attori:}\loggedusr
	\item \textbf{Precondizione:} l'utente si trova nella vista relativa ai filtri.
	\item \textbf{Scenario Principale:}
	\begin{enumerate}
		\item l'utente seleziona una o più città desiderate.
	\end{enumerate}
	\item \textbf{PostCondizione:}  l'utente si trova nella vista relativa ai filtri e visualizza le opzioni selezionate sullo schermo.
\end{itemize}

\subsection{UC7.10- Aggiunta di un filtro per l'anzianità}
\begin{itemize}
	\item \textbf{Attori:}\loggedusr
	\item \textbf{Precondizione:} l'utente si trova nella vista relativa ai filtri.
	\item \textbf{Scenario Principale:}
	\begin{enumerate}
		\item l'utente seleziona da un menù a tendina il livello di anzianità desiderato
	\end{enumerate}
	\item \textbf{PostCondizione:}  l'utente si trova nella vista relativa ai filtri e visualizza l'opzione selezionata sullo schermo.
\end{itemize}

\subsection{UC-8 Conferma dei filtri selezionati}
\begin{itemize}
\item \textbf{Attori:}\loggedusr
\item \textbf{Precondizione:} l'utente si trova nella vista relativa ai filtri.
\item \textbf{Scenario Principale:}
\begin{enumerate}
	\item l'utente clicca sul tasto "Applica filtri selezionati";
	\item l'utente viene reindirizzato alla vista di visualizzazione lista degli applicant.
\end{enumerate}
\item \textbf{PostCondizione:} l'utente si trova nella vista relativa alla visualizzazione degli applicant e visualizza li visualizza secondo i filtri selezionati;
\item \textbf{Estensioni:} l'utente visualizza un messaggio di errore relativo all'inserimento dei filtri (UC-8.1);
\end{itemize}

\subsection{UC-8.1 Visualizzazione errore sulla selezione dei filtri}
\begin{itemize}
\item \textbf{Attori:}\loggedusr
\item \textbf{Precondizione:} l'utente si trova nella vista relativa ai filtri e non ha selezionato congiuntamente una skill e un livello, ma solo uno dei due.
\item \textbf{Scenario Principale:}
\begin{enumerate}
	\item viene visualizzato un errore a schermo che invita l'utente a completare la selezione dei filtri.
\end{enumerate}
\item \textbf{PostCondizione:}l'utente si trova nella vista relativa ai filtri.
\end{itemize}


\subsection{UC-9 Rimozione dei filtri selezionati}
\begin{itemize}
\item \textbf{Attori:}\loggedusr
\item \textbf{Precondizione:} l'utente si trova nella vista relativa ai filtri
\item \textbf{Scenario Principale:}
\begin{enumerate}
	\item l'utente clicca sul tasto "Annulla";
	\item vengono rimossi tutti i flitri precedentemente selezionati
\end{enumerate}
\item \textbf{PostCondizione:} l'utente si trova nella vista relativa ai filtri.
\end{itemize}


\subsubsection{UC-10 Ordinamento degli\applicant secondo il nome}
\begin{itemize}
\item \textbf{Attori:}\loggedusr
\item \textbf{Precondizione:} l'utente si trova nella vista di visualizzazione lista degli applicant.
\item \textbf{Scenario Principale:}
\begin{enumerate}
	\item l'utente clicca sul \textit{header} della tabella riportante il campo "Nome"
	\item gli applicant vengono posti in ordine crescente secondo il nome.
\end{enumerate}
\item \textbf{PostCondizione:}  l'utente si trova nella vista di visualizzazione lista degli applicant.
\end{itemize}

\subsubsection{UC-11 Ordinamento degli\applicant secondo il cognome}
\begin{itemize}
	\item \textbf{Attori:}\loggedusr
	\item \textbf{Precondizione:} l'utente si trova nella vista di visualizzazione lista degli applicant.
	\item \textbf{Scenario Principale:}
	\begin{enumerate}
		\item l'utente clicca sul \textit{header} della tabella riportante il campo "Cognome"
		\item gli applicant vengono posti in ordine crescente secondo il cognome.
	\end{enumerate}
	\item \textbf{PostCondizione:}  l'utente si trova nella vista di visualizzazione lista degli applicant.
\end{itemize}

\subsection{UC-12 Visualizzazione vista di dettaglio di un\applicant}
\begin{itemize}
\item \textbf{Attori:}\loggedusr
\item \textbf{Precondizione:} l'utente si trova nella vista di visualizzazione lista degli applicant. 
\item \textbf{Scenario Principale:}
\begin{enumerate}
	\item l'utente clicca sul pulsante riportante una lente di ingrandimento relativa a un applicant
	\item l'utente viene reindirizzato alla vista di dettaglio di un applicant
	\item l'utente visualizza tramite la vista il dettaglio degli applicant
\end{enumerate}
\item \textbf{PostCondizione:} l'utente si trova nella vista di dettaglio di un applicant;
\end{itemize}

\subsection{UC-13 Visualizzazione dettaglio di un applicant }
\begin{itemize}
\item \textbf{Attori:} \loggedusr
\item \textbf{Precondizione:}  l'utente si trova nella vista di dettaglio di un applicant;
\item \textbf{Scenario Principale:}
\begin{enumerate}
	\item l'utente visualizza il nome dell'applicant;
	\item l'utente visualizza il cognome dell'applicant;
	\item l'utente visualizza il genere dell'applicant;
	\item l'utente visualizza la data di nascita dell'applicant;
	\item l'utente visualizza l'email dell'applicant;
	\item l'utente visualizza il numero di telefono dell'applicant;	
	\item l'utente visualizza l'indirizzo dell'applicant;
	\item l'utente visualizza la città residenza dell'applicant;
	\item l'utente visualizza il preavviso richiesto dall'applicant per una chiamata;
	\item l'utente visualizza il titolo di studio dell'applicant;
	\item l'utente visualizza il livello di anzianità dell'applicant;
	\item l'utente visualizza l'ambito di lavoro dell'applicant;
	\item l'utente visualizza se lo status dell'applicant è scartato o meno;
	
	\item l'utente visualizza le citta per cui l'applicant ha fornito disponibilità per il trasferimento;
	\item l'utente visualizza delle note generiche relative all'applicant;
	\item l'utente visualizza delle note relative al colloquio dell'applicant;
	\item l'utente visualizza delle note riguardanti lo status lavorativo dell'applicant;
	\item l'utente visualizza altre considerazioni aggiuntive relative all'applicant;
\end{enumerate}
\item \textbf{PostCondizione:}  l'utente si trova nella vista di dettaglio di un applicant;
\end{itemize}


\subsection{UC-14 Modifica di un dato di un applicant}
\begin{itemize}
\item \textbf{Attori:} Utente riconosciuto
\item \textbf{Precondizione:}  l'utente si trova nella vista di dettaglio di un applicant;
\item \textbf{Scenario Principale:}
\begin{enumerate}
	\item l'utente clicca su un pulsante posto a fianco di un campo dati di un applicant;
	\item l'utente visualizza il  widget di modifica per il campo dati
	relativo;
	\item l'utente inserisce le modifiche desiderate;
	\item l'utente clicca sul relativo pulsante precedentemente premuto;
\end{enumerate}
\item \textbf{PostCondizione:} 
\begin{enumerate}
	\item l'utente visualizza il campo dati modificato;
	\item l'utente si trova nella vista di dettaglio di un applicant.
\end{enumerate}

\end{itemize}

%%% MORE UC ON MODIFY FIELD

\subsection{UC-15 Salvataggio delle modifiche per un applicant}
\begin{itemize}
\item \textbf{Attori:} Utente riconosciuto
\item \textbf{Precondizione:} l'utente si trova nella vista di dettaglio di un applicant e visualizza il pulsante di salvataggio dopo aver modificato un campo.
\item \textbf{Scenario Principale:}
\begin{enumerate}
	\item l'utente clicca sul pulsante di salvataggio delle modifiche 
\end{enumerate}
\item \textbf{PostCondizione:} le modifiche vengono salvate e viene visualizzato un messaggio di successo.
\item \textbf{Estensioni:} viene visualizzato un messaggio di errore causato da un inserimento errato nei campi modificati (UC 15.1);
\end{itemize}



\subsection{UC-15.1 Visualizzazione messaggio di errore inserimento campo di un applicant}
\begin{itemize}
\item \textbf{Attori:} Utente riconosciuto
\item \textbf{Precondizione:}  l'utente si trova nella vista di dettaglio di un applicant e ha inserito un campo di un  applicant non rispettando i parametri imposti
\item \textbf{Scenario Principale:}
\begin{enumerate}
	\item viene visualizzato un messaggio di errore in corrispondenza del campo dati inserito erroneamente
\end{enumerate}
\item \textbf{PostCondizione:}  l'utente si trova nella vista di dettaglio di un applicant
\end{itemize}


%% MORE UC ERROR VISUALIZATION


\subsection{UC-16 Annullamento delle modifiche precendentemente inserite}
\begin{itemize}
\item \textbf{Attori:} Utente riconosciuto
\item \textbf{Precondizione:} l'utente si trova nella vista di dettaglio di un applicant e ha precedentemente confermato delle modifiche di alcuni dati relativi all'applicant
\item \textbf{Scenario Principale:}
\begin{enumerate}
	\item l'utente clicca sul pulsante di annullamento delle modifiche
\end{enumerate}
\item \textbf{PostCondizione:} l'utente visualizza la vista di dettaglio dell'applicant nello stato antecedente alla modifica.
\end{itemize}

\subsection{UC- 17 Ritorno alla lista degli applicant }
\begin{itemize}
\item \textbf{Attori:} Utente riconosciuto
\item \textbf{Precondizione:} l'utente si trova nella vista di dettaglio di un applicant
\item \textbf{Scenario Principale:}
\begin{enumerate}
	\item l'utente clicca sul pulsante \textit{back}
\end{enumerate}
\item \textbf{PostCondizione:} l'utente si trova nella vista di visualizzazione lista degli applicant.
\end{itemize}

\subsection{UC-18 Visualizzazione skillmatrix di un applicant}
\begin{itemize}
\item \textbf{Attori:} Utente riconosciuto
\item \textbf{Precondizione:}  l'utente si trova nella vista di dettaglio di un applicant
\item \textbf{Scenario Principale:}
\begin{enumerate}
	\item l'utente clicca sul pulsante relativo alla visualizzazione skillmatrix di un applicant
\end{enumerate}
\item \textbf{PostCondizione:} l'utente si trova nella vista relativa alla visualizzazione skillmatrix di un applicant.
\item \textbf{Estensioni:} l'utente si trova nella vista relativa all'inserimento skillmatrix di un applicant.
\end{itemize}